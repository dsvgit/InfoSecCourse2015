\documentclass[11pt, a4paper]{article}		% general format


%%%% Charset
\usepackage[utf8]{inputenc}					% use utf8					
\usepackage[russian]{babel}					% use russian font


%%%% Math
\usepackage{amsmath}						% Amer­i­can Math­e­mat­i­cal So­ci­ety (AMS) math fa­cil­i­ties
\usepackage{amsfonts}						% fonts from the AMS
\usepackage{amssymb}						% additional math symbols


%%%% Graphics
\usepackage{graphicx}


\author{Дедков Сергей}
\title{Отчет по лабораторной работе №4 :\\ Утилита для исследования сети и сканер портов nmap}
\date{2015}

%---------------------------------------------------------

\begin{document}
\maketitle
\tableofcontents
\newpage

%---------------------------------------------------------


\section{Цель работы}

Определить набор и версии сервисов запущенных на компьютере в диапазоне адресов.
Данная работа выполняется на ОС kali linux, используется утилита nmap.



%---------------------------------------------------------

\section{Ход работы}


%---------------------------------------------------------

\subsection{Провести поиск активных хостов}

Настройки сети: в нашей сети имеется всего 3 хоста.

\begin{itemize}
\item Windows 8 (192.168.150.1), основная ОС
\item kali linux (192.168.150.2)
\item metasploitable2 (192.168.150.3)
\end{itemize}

Выведем список хостов в подсети 192.168.150.0/24

Для этого воспользуемся командой \verb'nmap -sn 192.168.150.0/24'. (См. рисунок 1)

\begin{figure}[h!]
\centering
\includegraphics[scale=0.8]{res/hosts_search}
\caption{Поиск хостов}
\end{figure}

%---------------------------------------------------------

\subsection{Определить открытые порты}

Просканируем порты metasploitable2.

Для определения открытых портов достаточно просто ввести \verb'nmap 192.168.150.3' (сканируются порты до 1024). Или же воспользоваться опцией -p, например \verb'nmap -p "*" 192.168.150.3'. Данной командой просканируются все порты, если необходимо задать диапазон достаточно указать его вместо "*".
Результат на рисунке 2.

\begin{figure}[h!]
\centering
\includegraphics[scale=0.8]{res/ports_scan}
\caption{Поиск портов}
\end{figure}

%---------------------------------------------------------

\subsection{Определить версии сервисов}

Чтобы определить версии сервисов необходимо воспользоваться командой nmap с ключем sV следующим образом: \verb'nmap -sV 192.168.150.3'. 
Результат на рисунке 3.

\begin{figure}[h!]
\centering
\includegraphics[scale=0.8]{res/services_versions}
\caption{Определение версий сервисов}
\end{figure}

%---------------------------------------------------------

\subsection{Изучить файлы nmap-services, nmap-os-db, nmapservice-probes}

Рассмотрим файл nmap-services. Для этого введем команду 

vim /usr/share/nmap/nmap-services. 

Файл служит для быстрого поиска, напрмер с ключем -F. В файле в каждой строчке задаются сервисное название или сокращение, число порта и протокол, определенный разделом, частота порта мера того, как часто порт был найдет открытым во время сканирования. Пример файла можно увидеть на рисунке 4.


\begin{figure}[h!]
\centering
\includegraphics[scale=0.75]{res/nmap_services}
\caption{Файл nmap-services}
\end{figure}

Файл nmap-os-db содержит сотни примеров реакций ОС на nmap. Таким образом nmap определяет какая опреационная система установлена на удаленной машине. Для того чтобы узнать какая ОС установлена нужно запустить nmap с ключем -O. Содержимое файла представлено на рисунке 5.

\begin{figure}[h!]
\centering
\includegraphics[scale=0.75]{res/nmap_os_db}
\caption{Файл nmap-os-db}
\end{figure}


nmap-service-probes — это простой текстовый файл состоящий из строк, в котором хнаняться тесты и сигнатуры подсистем определений версий. Строки, начинающиеся с символа "решетки" (\#) воспринимаются как комментарии и игнорируются обработчиком. Пустые строки также не обрабатываются. 


Синтаксис: 
\begin{itemize}
\item Probe <protocol> <probename> <probesendstring> - директива probe (тест) - указывает nmap, какие данные отправлять в процессе определения служб
\item match <service> <pattern> <productname> <version> <device> <h?????> <info> <OS> - указывает nmap на то, как точно определить службу, используя полученный ответ на запрос, отправленный предыдущей директивой probe. Эта директива используется в случае, когда полученный ответ полностью совпадает с шаблоном. При этом тестирование порта считается законченным, а при помощи дополнительных спецификаторов nmap строит отчет о названии приложения, номере версии и дополнительной информации, полученной в ходе проверки
\item softmatch <service> <pattern> <productname> <version> <device> <h?????> <info> <OS> - имеет аналогичный формат директиве match. Основное отличие заключается в том, что после совпадения принятого ответа с одним из шаблонов softmatch, тестирование будет продолжено с использованием только тех тестов, которые относятся к определенной шаблоном службе. Тестирование порта будет идти до тех пор, пока не будет найдено строгое соответствие (match) или не закончатся все тесты для данной службы
\item ports <portlist> - группирует порты, которые обычно закрепляются за идентифицируемой данным тестом службой
\item sslports <sslportlist> - аналогична директиве ports, только эта директива указывает порты, обычно используемые совместно с SSL
\item totalwaitms <milliseconds> - редко используемая, т.к. указывает сколько времени (в миллисекундах) необходимо ждать ответ, прежде чем прекратить тест службы
\end{itemize}

%---------------------------------------------------------

\subsection{Добавить новую сигнатуру службы в файл nmap-service-probes (для этого создать минимальный tcp server, добиться, чтобы при сканировании nmap указывал для него название и версию)}

Напишем простой tcp-сервер, который просто ждет подключения клиента и отправляет ему сообщение. В файл nmap-service-probes добавим следующую строку: 

\verb"match tcp-server m|^111| v/1.0.X/ p/Dedkov S.V./ i/It's works /"

Код сервера:

\begin{verbatim}
using System;
using System.Collections.Generic;
using System.Linq;
using System.Text;
using System.IO;
using System.Net;
using System.Net.Sockets;
using System.Threading;

namespace ExampleTcpListener_Console
{
    class ExampleTcpListener
    {
        static void Main(string[] args)
        {
            TcpListener server = null;
            try
            {
                int MaxThreadsCount = Environment.ProcessorCount * 4;
                Console.WriteLine(MaxThreadsCount.ToString());
                ThreadPool.SetMaxThreads(MaxThreadsCount, MaxThreadsCount);
                ThreadPool.SetMinThreads(2, 2);

                Int32 port = 9596;
                IPAddress localAddr = IPAddress.Parse("192.168.137.1");
                int counter = 0;
                server = new TcpListener(localAddr, port);

                server.Start();

                while (true)
                {

                    Console.Write("\nWaiting for a connection... ");

                    ThreadPool.QueueUserWorkItem(ObrabotkaZaprosa, server.AcceptTcpClient());
                    counter++;
                    Console.Write("\nConnection №" + counter.ToString() + "!");

                }
            }
            catch (SocketException e)
            {
                Console.WriteLine("SocketException: {0}", e);
            }
            finally
            {
                server.Stop();
            }
            
            Console.WriteLine("\nHit enter to continue...");
            Console.Read();
        }

        static void ObrabotkaZaprosa(object client_obj)
        {
            Byte[] bytes = new Byte[256];
            String data = null;

            TcpClient client = client_obj as TcpClient;

            data = null;

            NetworkStream stream = client.GetStream();

            int i;

            data = "111";
            byte[] msg = System.Text.Encoding.ASCII.GetBytes(data);
            stream.Write(msg, 0, msg.Length);

            client.Close();
        }
    }
}
\end{verbatim}
  
Таким образом теперь nmap знает, что если при пустом запросе с сервера прихоит строка 111, значит нужно выводить информацию которая указана на рисунке 6.

\begin{figure}[h!]
\centering
\includegraphics[scale=0.8]{res/server_response}
\caption{Вывод информации о сервисе}
\end{figure}



%---------------------------------------------------------

\subsection{Сохранить выводы утилиты в формате xml}

Для того, чтобы вывести данные в xml файл достаточно вызвать команду nmap с ключем -oX и указать имя файла. Например: 

\verb'namp -sn -oX output.xml 192.168.150.1'

Результат можно увидеть на рисунке 7:

\begin{figure}[h!]
\centering
\includegraphics[scale=0.8]{res/xml_output}
\caption{output.xml}
\end{figure}

%---------------------------------------------------------

\subsection{Исследовать различные этапы и режимы работы nmap с использованием утилиты Wireshark}

Wireshark — это достаточно известный инструмент для захвата и анализа сетевого трафика, фактически стандарт как для образования, так и для траблшутинга. 
Wireshark работает с подавляющим большинством известных протоколов, имеет понятный и логичный графический интерфейс на основе GTK+ и мощнейшую систему фильтров.
Кроссплатформенный, работает в таких ОС как Linux, Solaris, FreeBSD, NetBSD, OpenBSD, Mac OS X, и, естественно, Windows. Распространяется под лицензией GNU GPL v2. Доступен бесплатно на сайте wireshark.org.

Далее продемонстрируем простую работу с wireshark. При запуске wireshark предложит выбрать интерфейс и начать сканировать его траффик(см. рисунок 8). Выберем интерфейс eth0

\begin{figure}[h!]
\centering
\includegraphics[scale=0.8]{res/wireshark_int}
\caption{wireshark выбор интерфейса и старт}
\end{figure}

Wireshark по-умолчанию выводит все пакеты, которые проходят через интерфейс eth0. Все пакеты просматривать неудобно, поэтому мы будем пользоваться фильтрами. Поставим фильтр по протоколу icmp, ip отправителя и получателя: 

\verb'ip.src == 192.168.150.2 and ip.dst == 192.168.150.3 and icmp'. 

Введем команду для определения ОС в подсети 192.168.150.0/24 

\verb'nmap -O 192.168.150.0/24'.

Результат на рисунке 9. Увидим три пакета

\begin{figure}[h!]
\centering
\includegraphics[scale=0.8]{res/wireshark_scan_icmp}
\caption{wireshark scan}
\end{figure}

По каждому запросу можно увидеть дополнительую информацию кликнув на нем, как, например, на рисунке 10.

\begin{figure}[h!]
\centering
\includegraphics[scale=0.8]{res/wireshark_info}
\caption{wireshark info}
\end{figure}

%---------------------------------------------------------

\subsection{Просканировать виртуальную машину Metasploitable2 используя nmap\_db из состава metasploit-framework}

Для работы с metasploit потребуется СУБД postgresql. Для того, чтобы ее запустить воспользуемся командой 

\verb'service postgressql start'

Затем запустим msf консоль командой \verb'msfconsole'.

В данной консоли осуществляется работа с metasplit. Проверим соединение с БД:

\verb'db_status'

Результат на рисунке 11.

\begin{figure}[h!]
\centering
\includegraphics[scale=0.8]{res/db_status}
\caption{db\_status}
\end{figure}

Таблицы БД: hosts, services, vulns, loot и notes. В каждой храниться соответствующая информация.
Для заполнения этих таблиц автоматизированно, можно использовать db\_nmap. Так же можно использовать какую-то любую утилиту для сканирования, экспортировать результаты её работы в XML-файл, а потом - импортировать его в метасплойт. Это можно сделать, использовав db\_import внутри меню метасплойт.

Выполним команду: \verb'db_nmap -sV 192.168.150.0/24'

Вывод будет такой же как и при использовании команы nmap. Теперь после того как была просканирована наша сеть можно посмотреть, что было записано в БД: \verb'hosts', \verb'services'. Результаты на рисунках 12 и 13.

\begin{figure}[h!]
\centering
\includegraphics[scale=0.8]{res/msf_hosts}
\caption{Таблица хостов}
\end{figure}

\begin{figure}[h!]
\centering
\includegraphics[scale=0.8]{res/msf_services}
\caption{Таблица сервисов}
\end{figure}

%---------------------------------------------------------

\subsection{Выбрать пять записей из файла nmap-service-probes и описать их работу. Выбрать один скрипт из состава Nmap и описать его работу}

\begin{itemize}
\item Первая запись

Возьмем самую первую запсись probe - эта запись теста с отправкой null-запроса. В данной записи будет отправляться пустой запрос по протоколу TCP. С ожиданием ответа в 6 секунд(директива totalwaitms).

\begin{figure}[h!]
\centering
\includegraphics[scale=0.8]{res/probe_1}
\caption{Запись 1}
\end{figure}

\item Вторая запись

Второй записью рассмотрим match после probe null-запроса. Если пользователь укажет ключ -sV при использовании nmap и после отправки нулевого теста с сервера приедет выражение подходящее под mSxf5xc6x1a тогда в колонке SERVICE при выводе информации он увидит наименование сервиса 1c-server, а в олонке VERSION 1C:Enterprise business management server.

\begin{figure}[h!]
\centering
\includegraphics[scale=0.8]{res/probe_2}
\caption{Запись 2}
\end{figure}

\item Третья запись

В данной запись на сервер отправляется запрос по протоколу TCP в котором передается информация. Так же указан список портов и ssl-портов, по которым нужно осуществлять сканирование.

\begin{figure}[h!]
\centering
\includegraphics[scale=0.8]{res/probe_3}
\caption{Запись 3}
\end{figure}

\item Четвертая запись

Здесь отправляется запрос по протоколу UDP для проверки RPC.

\begin{figure}[h!]
\centering
\includegraphics[scale=0.8]{res/probe_4}
\caption{Запись 4}
\end{figure}

\item Пятая запись

Здесь отправляется запрос по протоколу UDP для проверки sql, через порт 1434.


\begin{figure}[h!]
\centering
\includegraphics[scale=0.8]{res/probe_5}
\caption{Запись 5}
\end{figure}




\item Скрипт

Рассмотрим следующий скрипт \verb'skypev2-version.nse'. Его можно найти в папке с отчетом.

Первым делом в нем объявлены переменные импортированные из библиотек:

\begin{verbatim}
local comm = require "comm"
local nmap = require "nmap"
local shortport = require "shortport"
local string = require "string"
local U = require "lpeg-utility"
\end{verbatim}

Далее следует описание скрипта в переменной \verb'description':

\begin{verbatim}
description = [[
Detects the Skype version 2 service.
]]
\end{verbatim}

Далее описание в комментарии в формате NSEDoc.

\begin{verbatim}
---
-- @output
-- PORT   STATE SERVICE VERSION
-- 80/tcp open  skype2  Skype
\end{verbatim}

Далее имя автора, лицензия, категория:

\begin{verbatim}
author = "Brandon Enright"
license = "Same as Nmap--See http://nmap.org/book/man-legal.html"
categories = {"version"}
\end{verbatim}

Далее правило порта(вместо него можно задавать правило хоста) - функция возвращающая true или false. Если возвращает true, то выполняется функция заданная в перемнной action. В данном случае из кода понятно, в каких случая выполняется это условие.
 
\begin{verbatim}
portrule = function(host, port)
  return (port.number == 80 or port.number == 443 or
    port.service == nil or port.service == "" or
    port.service == "unknown") -- условия по портам
  and port.protocol == "tcp" and port.state == "open" -- условия по протоколу
  and port.version.name_confidence < 10 -- доверие
  and not(shortport.port_is_excluded(port.number,port.protocol)) -- порт не исключен
  and nmap.version_intensity() >= 7 -- версия интенсивности
end
\end{verbatim}

Далее приведен код функции action:

\begin{verbatim}
action = function(host, port)
  local result, rand
  -- Did the service engine already do the hard work?
  if port.version and port.version.service_fp then
    -- Probes sent, replies received, but no match.
    result = U.get_response(port.version.service_fp, "GetRequest")
    -- Loop through the ASCII probes most likely to receive random response
    -- from Skype. Others will also recieve this response, but are harder to
    -- distinguish from an echo service.
    for _, p in ipairs({"HTTPOptions", "RTSPRequest"}) do
      rand = U.get_response(port.version.service_fp, p)
      if rand then
        break
      end
    end
  end
  local status
  if not result then
    -- Have to send the probe ourselves.
    status, result = comm.exchange(host, port,
      "GET / HTTP/1.0\r\n\r\n", {bytes=26})

    if (not status) then
      return nil
    end
  end

  if (result ~= "HTTP/1.0 404 Not Found\r\n\r\n") then
    return
  end

  -- So far so good, now see if we get random data for another request
  if not rand then
    status, rand = comm.exchange(host, port,
      "random data\r\n\r\n", {bytes=15})

    if (not status) then
      return
    end
  end

  if string.match(rand, "[^%s!-~].*[^%s!-~].*[^%s!-~]") then
    -- Detected
    port.version.name = "skype2"
    port.version.product = "Skype"
    nmap.set_port_version(host, port)
    return
  end
  return
end
\end{verbatim}

После прохождения всех проверок, обозначается версия и продукт(после комментария Detected).

Для использования скрипта нужно ввести либо ключ -sC, либо --script=<имя\_скрипта>.

Для примера была просканирована основная ОС. Двумя способами. Результаты можно увидеть на рисунках 19 и 20.

\begin{figure}[h!]
\centering
\includegraphics[scale=0.8]{res/script_1}
\caption{Результаты сканирования без использования скрипта}
\end{figure}

\begin{figure}[h!]
\centering
\includegraphics[scale=0.8]{res/script_2}
\caption{Результаты сканирования с использованием скрипта}
\end{figure}

\end{itemize}

%---------------------------------------------------------

\section{Вывод}

В ходе проделанной работы были изучены основы работы с nmap, немного изучен сниффер wireshark, а так же запись в БД metasploit посредством команды db\_nmap. Ранее я не был знаком с подобными программами, поэтому не с чем сравнивать. В общем интересно было посканировать хосты и порты. Понравилось так же что можно писать собственные скрипты, но для того, чтобы подробно с ними разобраться нужно больше времени. 


\end{document}